\documentclass[12pt]{article}

\usepackage[margin=1in]{geometry}
\usepackage{amsmath,amsthm,amssymb}
\usepackage{mathtools}
\usepackage{multicol}
\usepackage{textcomp}
\usepackage{float}
\usepackage{longtable}

\newcommand{\N}{\mathbb{N}}
\newcommand{\Z}{\mathbb{Z}}
\newcommand\aug{\fboxsep=-\fboxrule\!\!\!\fbox{\strut}\!\!\!}

\newenvironment{theorem}[2][Theorem]{\begin{trivlist}
\item[\hskip \labelsep {\bfseries #1}\hskip \labelsep {\bfseries #2.}]}{\end{trivlist}}
\newenvironment{lemma}[2][Lemma]{\begin{trivlist}
\item[\hskip \labelsep {\bfseries #1}\hskip \labelsep {\bfseries #2.}]}{\end{trivlist}}
\newenvironment{exercise}[2][Exercise]{\begin{trivlist}
\item[\hskip \labelsep {\bfseries #1}\hskip \labelsep {\bfseries #2.}]}{\end{trivlist}}
\newenvironment{reflection}[2][Reflection]{\begin{trivlist}
\item[\hskip \labelsep {\bfseries #1}\hskip \labelsep {\bfseries #2.}]}{\end{trivlist}}
\newenvironment{proposition}[2][Proposition]{\begin{trivlist}
\item[\hskip \labelsep {\bfseries #1}\hskip \labelsep {\bfseries #2.}]}{\end{trivlist}}
\newenvironment{corollary}[2][Corollary]{\begin{trivlist}
\item[\hskip \labelsep {\bfseries #1}\hskip \labelsep {\bfseries #2.}]}{\end{trivlist}}

\begin{document}

\title{TUTORIAL 5}
\author{Timothée Guédon \& Tristan Glatard\\
COMP 361 Numerical Methods}
\date{October 11, 2019}
\maketitle

\section{Exercises for today}

\begin{exercise}{1}
Use Newton-Raphson method to compute $\sqrt[3]{75}$ with four significant figure accuracy. Start at $x = 5$.
\end{exercise}

\begin{exercise}{2}
The smallest positive, non-zero root of $cosh(x)cos(x)-1=0$ lies in the interval
(4, 5). Compute this root using the \textit{regula falsi} method.
\end{exercise}

\begin{exercise}{3}
Solve Exercise 2 by the Newton-Raphson method, starting from $x=5$.
\end{exercise}

\begin{exercise}{4}
A root of the equation $tan(x)-tanh(x)=0$ lies in (7.0, 7.4). Find this root with three
decimal place accuracy by the method of bisection.
\end{exercise}

\begin{exercise}{5}
Use linear regression to find the line that fits the following data and determine the standard deviation:\\
\begin{table}[h]
  \centering
  \begin{tabular}{|c|c|c|c|c|c|}
    \hline
    x & -1.0 & -0.5 & 0 & 0.5 & 1.0 \\ \hline
    y & -1.00 & -0.55 & 0.00 & 0.45 & 1.00 \\ \hline
  \end{tabular}
\end{table}

\end{exercise}

\break

\section{Solutions}

%-------------------------------------------------------------------------------------------------------
\subsection{Exercise 1}
%-------------------------------------------------------------------------------------------------------

The problem is equivalent to finding the root of $f(x) = x^3-75 = 0$. \\
The Newton-Raphson formula gives us: \\
\begin{align}
\notag
x \gets x - \Delta x
\end{align}
where
\begin{align}
\Delta x = \frac{f(x)}{f\prime(x)} = \frac{x^3-75}{3x^2} = \frac{x}{3} - \frac{25}{x^2}
\end{align}

\noindent Starting at $x = 5$, \\

\noindent Iteration 1 \\
\begin{itemize}
  \item $\Delta x = \frac{5}{3} - \frac{25}{5^2} = \frac{2}{3} \approx 0.6667 >  10e^{-4}$ \\
\end{itemize}

\noindent Iteration 2 \\
\begin{itemize}
  \item $x = 5 - \frac{2}{3} = \frac{13}{3}$ \\
  \item $\Delta x = \frac{\frac{13}{3}}{3} - \frac{25}{(\frac{13}{3})^2} = \frac{172}{1521} \approx 0.113 > 10e^{-4}$ \\
\end{itemize}

\noindent Iteration 3 \\
\begin{itemize}
  \item $x = \frac{13}{3} - \frac{172}{1521} = \frac{6419}{1521}$ \\
  \item $\Delta x = \frac{\frac{6419}{1521}}{3} - \frac{25}{(\frac{6419}{1521})^2} = \frac{6419}{1521*3} - \frac{25 * 1521^2}{6419^2} \approx 1.40675 - 1.403665 = 0.003085 > 10e^{-4} $\\
\end{itemize}

\noindent Iteration 4 \\
\begin{itemize}
  \item $x = \frac{6419}{1521} - 0.003085 \approx 4.2171648 $ \\
  \item $\Delta x = \frac{4.2171648}{3} - \frac{25}{4.2171648^2} = 0.000001473 < 10e^{-4}$ \\
\end{itemize}

\noindent We can then find our final $x$:
$ x = 4.2171648 - 0.000001473 = 4.217163327 $ \\

\noindent If we compute $\sqrt[3]{75}$ with a calculator \textit{(fedora linux)} we find: $4.217163327$ \\
\noindent We can also verify our result as follows: $4.217163327^3 - 75 = 0.000000026$

%-------------------------------------------------------------------------------------------------------
\subsection{Exercise 2}
%-------------------------------------------------------------------------------------------------------

We estimate the root with
\begin{align}
x_3 = x_2 - f_2 \frac{x_2-x_1}{f_2-f_1} \notag
\end{align}

At each iteration, we evaluate \(x_3\) and \(f(x_3)\). If \(f(x_3)\) has the same sign as \(f(x_1)\), replace \(x_1\) by \(x_3\). Otherwise replace \(x_2\) by \(x_3\).
We stop when \(|x_3^{(k+1)}-x_3^{(k)}| \leq \epsilon = 10^{-4}\) where \(x_3^{(k)}\)is the approximation of root at iteration \textit{k}. The table below lists the calculation of each iteration.
\begin{table}[H]
\centering
\begin{tabular}{|c|r|r|r|r|r|r|}
\hline
\textbf{Iteration} & \multicolumn{1}{c|}{\textbf{\(x_1\)}} & \multicolumn{1}{c|}{\textbf{\(x_2\)}} & \multicolumn{1}{c|}{\textbf{\(x_3\)}} & \multicolumn{1}{c|}{\textbf{\(f(x_1)\)}} & \multicolumn{1}{c|}{\textbf{\(f(x_2)\)}} & \multicolumn{1}{c|}{\textbf{\(f(x_3)\)}} \\ \hline
\textit{1} & 4.00000 & 5 & 4.48457 & -18.8499 & 20.0506 & -11.0111 \\ \hline
\textit{2} & 4.48457 & 5 & 4.66728 & -11.0111 & 20.0506 & -3.3992 \\ \hline
\textit{3} & 4.66728 & 5 & 4.71551 & -3.3992 & 20.0506 & -0.8256 \\ \hline
\textit{4} & 4.71551 & 5 & 4.72676 & -0.8256 & 20.0506 & -0.1883 \\ \hline
\textit{5} & 4.72676 & 5 & 4.72931 & -0.1883 & 20.0506 & -0.0423 \\ \hline
\textit{6} & 4.72931 & 5 & 4.72988 & -0.0423 & 20.0506 & -0.0095 \\ \hline
\textit{7} & 4.72988 & 5 & 4.73000 & -0.0095 & 20.0506 & -0.0021 \\ \hline
\textit{8} & 4.73000 & 5 & 4.73003 & -0.0021 & 20.0506 & -0.0005 \\ \hline
\end{tabular}
\end{table}

The estimated root is \textbf{4.73003} .

%-------------------------------------------------------------------------------------------------------
\subsection{Exercise 3}
%-------------------------------------------------------------------------------------------------------

$$f(x) = cosh(x)cos(x)-1=0$$

For the Newton-Raphson method, we need the first derivative of f(x):
\begin{align}
f^\prime(x) = -cosh(x)sin(x) + sinh(x)cos(x) \notag
\end{align}

Steps of Newton-Raphson are in the following table, stopping condition is $|\Delta| \leq \epsilon = 10e^{-4}$

\begin{table}[H]
\centering
\begin{tabular}{|c|r|r|}
\hline
\textbf{Iteration} & \multicolumn{1}{c|}{\textbf{x}} & \multicolumn{1}{c|}{\textbf{\(\Delta\)}} \\ \hline
\textit{1} & 5.00000 & 0.21744 \\ \hline
\textit{2} & 4.78256 & 0.04998 \\ \hline
\textit{3} & 4.73258 & 0.00253 \\ \hline
\textit{4} & 4.73005 & 0.000009 \\ \hline
\end{tabular}
\end{table}

The approximated root is  \textbf{4.73005}.

\break

%-------------------------------------------------------------------------------------------------------
\subsection{Exercise 4}
%-------------------------------------------------------------------------------------------------------

We have the following table.

\begin{table}[H]
\centering
\begin{tabular}{|c|r|r|r|r|r|r|}
\hline
\textbf{Iteration} & \multicolumn{1}{c|}{\textbf{\(x_1\)}} & \multicolumn{1}{c|}{\textbf{\(x_2\)}} & \multicolumn{1}{c|}{\textbf{\(x_3\)}} & \multicolumn{1}{c|}{\textbf{\(f(x_1)\)}} & \multicolumn{1}{c|}{\textbf{\(f(x_2)\)}} & \multicolumn{1}{c|}{\textbf{\(f(x_3)\)}} \\ \hline
\textit{1} & 7.0000 & 7.4000 & 7.2000 & -0.12855 & 1.04928 & 0.30462 \\ \hline
\textit{2} & 7.0000 & 7.2000 & 7.1000 & -0.12855 & 0.30462 & 0.06489 \\ \hline
\textit{3} & 7.0000 & 7.1000 & 7.0500 & -0.12855 & 0.06489 & -0.03649 \\ \hline
\textit{4} & 7.0500 & 7.1000 & 7.0750 & -0.03649 & 0.06489 & 0.01292 \\ \hline
\textit{5} & 7.0500 & 7.0750 & 7.0625 & -0.03649 & 0.01292 & -0.01209 \\ \hline
\textit{6} & 7.0625 & 7.0750 & 7.0688 & -0.01209 & 0.01292 & 0.00033 \\ \hline
\textit{7} & 7.0625 & 7.0688 & 7.0656 & -0.01209 & 0.00033 & -0.00590 \\ \hline
\textit{8} & 7.0656 & 7.0688 & 7.0672 & -0.00590 & 0.00033 & -0.00279 \\ \hline
\textit{9} & 7.0672 & 7.0688 & 7.0680 & -0.00279 & 0.00033 & -0.00123 \\ \hline
\textit{10} & 7.0680 & 7.0688 & 7.0684 & -0.00123 & 0.00033 & -0.00045 \\ \hline
\end{tabular}
\end{table}

\noindent We stopped at iteration 10 when \(|x_1-x_2| \leq \epsilon = 10^{-3}\) and have the estimated root of \textbf{7.0684}.


%-------------------------------------------------------------------------------------------------------
\subsection{Exercise 5}
%-------------------------------------------------------------------------------------------------------

With linear regression, we try to fit the data into a line of form
\begin{align}
  f(x)=a+bx
\end{align}

where parameters a, b are given as
\begin{align}
  b&=\frac{\sum\limits_{i=0}^{n} y_i(x_i-\overline{x})}{\sum\limits_{i=0}^{n} x_i(x_i-\overline{x})}\\
  a&=\overline{y}-\overline{x}b
\end{align}

From the data:
\begin{align}
  \notag
  \overline{x}&=\frac{\sum\limits_{i=0}^{n} x_i}{n+1} = \frac{-1.0-0.5+0+0.5+1.0}{5} = 0 \\
  \notag
  \overline{y}&=\frac{\sum\limits_{i=0}^{n} y_i}{n+1}= \frac{-1.00 - 0.55 + 0.00 + 0.45 + 1.00}{5} = -0.02
\end{align}

Apply to (2) and (3)
\begin{align}
  \notag
  b&= \frac{(-1.00)(-1.0) + (-0.55)(-0.5) + 0.00*0 + 0.45*0.5 + 1.00*1.0}{(-1.0)(-1.0)+(-0.5)(-0.5)+0*0+0.5*0.5+1.0*1.0} = \frac{2.5}{2.5} = 1\\
  \notag
  a&=-0.02-0*1=-0.02
\end{align}

Then (1) becomes
\begin{align}
  \notag
  f(x)=-0.02 + x
\end{align}

We start the evaluation of the standard deviation by computing the residuals:

\begin{table}[h]
  \centering
  \begin{tabular}{|c|c|c|c|c|c|}
    \hline
    x & -1.0 & -0.5 & 0 & 0.5 & 1.0 \\ \hline
    y & -1.00 & -0.55 & 0.00 & 0.45 & 1.00 \\ \hline
    f(x) & -1.02 & -0.52 & -0.02 & 0.48 & 0.98 \\ \hline
    y - f(x) & 0.02 & -0.03 & 0.02 & -0.03 & 0.02 \\ \hline
  \end{tabular}
\end{table}

The sum of the squares of the residuals is

\begin{align}
  \notag
  S&=\sum\limits_{i=0}^4 [y_i-f(x_i)]^2\\
  \notag
  &=(0.02)^2 + (-0.03)^2 + (0.02)^2 + (-0.03)^2 + (0.02)^2\\
  \notag
  &=0.003
\end{align}

The standard deviation is given as

\begin{align}
  \notag
  \sigma=\sqrt[]{\frac{S}{n-m}}
\end{align}

where $S$ is the sum of the squares of the residuals calculated above, $n + 1$ is the number of data points (i.e., $n = 4$), $m$ is the degree of the polynomial (1 in this case, since we fit linear regression)\\

Finally,

\begin{align}
  \notag
  \sigma=\sqrt[]{\frac{0.003}{4-1}} \approx 0.0316
\end{align}


\end{document}
