\documentclass[12pt]{article}

\usepackage[margin=1in]{geometry}
\usepackage{amsmath,amsthm,amssymb}
\usepackage{mathtools}
\usepackage{multicol}
\usepackage{textcomp}
\usepackage{float}
\usepackage{longtable}

\newcommand{\N}{\mathbb{N}}
\newcommand{\Z}{\mathbb{Z}}
\newcommand\aug{\fboxsep=-\fboxrule\!\!\!\fbox{\strut}\!\!\!}

\newenvironment{theorem}[2][Theorem]{\begin{trivlist}
\item[\hskip \labelsep {\bfseries #1}\hskip \labelsep {\bfseries #2.}]}{\end{trivlist}}
\newenvironment{lemma}[2][Lemma]{\begin{trivlist}
\item[\hskip \labelsep {\bfseries #1}\hskip \labelsep {\bfseries #2.}]}{\end{trivlist}}
\newenvironment{exercise}[2][Exercise]{\begin{trivlist}
\item[\hskip \labelsep {\bfseries #1}\hskip \labelsep {\bfseries #2.}]}{\end{trivlist}}
\newenvironment{reflection}[2][Reflection]{\begin{trivlist}
\item[\hskip \labelsep {\bfseries #1}\hskip \labelsep {\bfseries #2.}]}{\end{trivlist}}
\newenvironment{proposition}[2][Proposition]{\begin{trivlist}
\item[\hskip \labelsep {\bfseries #1}\hskip \labelsep {\bfseries #2.}]}{\end{trivlist}}
\newenvironment{corollary}[2][Corollary]{\begin{trivlist}
\item[\hskip \labelsep {\bfseries #1}\hskip \labelsep {\bfseries #2.}]}{\end{trivlist}}

\begin{document}

\title{TUTORIAL 2}
\author{Timothée Guédon \& Tristan Glatard\\
COMP 361 Numerical Methods}
\date{September 20, 2019}
\maketitle

\section{Exercises for today}

\begin{exercise}{1}
Invert the following matrix using Gauss elimination or Gauss-Jordan elimination:
\begin{center}
\textbf{B}=
$
\begin{bmatrix}
2 & -2 & 1 \\
1 & 0 & -1 \\
4 & 1 & 1
\end{bmatrix}
$
\end{center}
\end{exercise}

\begin{exercise}{2}
Find Doolittle's LU decomposition of \textbf{B}:\\
\begin{center}
\textbf{B}=
$
\begin{bmatrix}
2 & -2 & 1 \\
1 & 0 & -1 \\
4 & 1 & 1
\end{bmatrix}$
\end{center}
Then solve $Bx=b$ for $b = \begin{bmatrix}
  1 \\ 2 \\ 0
\end{bmatrix}$
\end{exercise}

\begin{exercise}{3}
Find the Cholesky decomposition of \textbf{A}: \\
\begin{center}
\textbf{A}=
$
\begin{bmatrix}
1 & 1 & 1 \\
1 & 2 & 2 \\
1 & 2 & 3
\end{bmatrix}
$
\end{center}
\end{exercise}

\begin{exercise}{4} (As a training)
Invert the following matrix:
\begin{center}
\textbf{D}=
$
\begin{bmatrix}
3 & 1 & 2 \\
1 & 1 & 0 \\
5 & 8 & 9
\end{bmatrix}
$
\end{center}
\end{exercise}

\break

\section{Solutions}

%-------------------------------------------------------------------------------------------------------
\subsection{Exercise 1}
%-------------------------------------------------------------------------------------------------------

\textbf{Gauss elimination} \\

\begin{center}
\textbf{[$B \vert I$]}=
$\begin{bmatrix}
2 & -2 & 1 & \aug & 1 & 0 & 0 \\
1 & 0 & -1 & \aug & 0 & 1 & 0 \\
4 & 1 & 1 & \aug & 0 & 0 & 1 \\
\end{bmatrix}
$
\end{center}

\noindent $L_1 \gets L_3$ \\
$L_3 \gets L_1$ \\
\begin{center}
$\begin{bmatrix}
4 & 1 & 1 & \aug & 0 & 0 & 1 \\
1 & 0 & -1 & \aug & 0 & 1 & 0 \\
2 & -2 & 1 & \aug & 1 & 0 & 0 \\
\end{bmatrix}$
\end{center}

\noindent $L_1 \gets L_1/4$ \\
\begin{center}
$\begin{bmatrix}
1 & \frac{1}{4} & \frac{1}{4} & \aug & 0 & 0 & \frac{1}{4} \\
1 & 0 & -1 & \aug & 0 & 1 & 0 \\
2 & -2 & 1 & \aug & 1 & 0 & 0 \\
\end{bmatrix}$
\end{center}

\noindent $L_2 \gets L_2 - L_1$ \\
$L_3 \gets L_3 - 2 * L_1$ \\
\begin{center}
$\begin{bmatrix}
1 & \frac{1}{4} & \frac{1}{4} & \aug & 0 & 0 & \frac{1}{4} \\
0 & \frac{-1}{4} & \frac{-5}{4} & \aug & 0 & 1 & \frac{-1}{4} \\
0 & \frac{-10}{4} & \frac{2}{4} & \aug & 1 & 0 & \frac{-2}{4} \\
\end{bmatrix}$
\end{center}

\noindent $L_3 \gets L_3 - 10 * L_2$ \\
\begin{center}
$\begin{bmatrix}
1 & \frac{1}{4} & \frac{1}{4} & \aug & 0 & 0 & \frac{1}{4} \\
0 & \frac{-1}{4} & \frac{-5}{4} & \aug & 0 & 1 & \frac{-1}{4} \\
0 & 0 & 13 & \aug & 1 & -10 & \frac{8}{4} \\
\end{bmatrix}$
\end{center}


\noindent $L_3 \gets L_3/13$ \\
\begin{center}
$\begin{bmatrix}
1 & \frac{1}{4} & \frac{1}{4} & \aug & 0 & 0 & \frac{1}{4} \\
0 & \frac{-1}{4} & \frac{-5}{4} & \aug & 0 & 1 & \frac{-1}{4} \\
0 & 0 & 1 & \aug & \frac{1}{13} & \frac{-10}{13} & \frac{2}{13} \\
\end{bmatrix}$
\end{center}

\noindent For the Gauss elimination method, we stop here and we solve the following system of equations by backward substitution for each vector $b$ in the augmented matrix: \\
$$\begin{cases}
  a + \frac{1}{4}b + \frac{1}{4}c = b_1 \\
  \frac{-1}{4}b + \frac{-5}{4}c = b_2 \\
  c = b_3
\end{cases}$$ \\

with $$ b = \begin{bmatrix}
  0 \\
  0 \\
  \frac{1}{13}
\end{bmatrix} b = \begin{bmatrix}
  0 \\
  1 \\
  \frac{-10}{13}
\end{bmatrix} b = \begin{bmatrix}
  \frac{1}{4} \\
  \frac{-1}{4} \\
  \frac{2}{13}
\end{bmatrix} $$ \\

You should then find: \\

$$B^{-1} = \frac{1}{52}
\begin{bmatrix}
4 & 12 & 8 & \\
-20 & -8 & 12 \\
4 & -40 & 8
\end{bmatrix}$$ \\

\noindent \textbf{Gauss-Jordan elimination} \\
Gauss-Jordan elimination is just the Gauss elimination taken to its limit: When $U$ is found, you can continue the elimination on $U$ with pivots from $U_{N, N}$ to $U_{N-1, N-1}$.\\

\noindent $L_2 \gets L_2 + \frac{5}{4}L_3$ \\
$L_1 \gets L_1 - \frac{1}{4}L_3$ \\
\begin{center}
$\begin{bmatrix}
1 & \frac{1}{4} & 0 & \aug & \frac{-1}{52} & \frac{10}{52} & \frac{11}{52} \\
0 & \frac{-1}{4} & 0 & \aug & \frac{5}{52} & \frac{2}{52} & \frac{-3}{52} \\
0 & 0 & 1 & \aug & \frac{1}{13} & \frac{-10}{13} & \frac{2}{13} \\
\end{bmatrix}$
\end{center}

\noindent $L_1 \gets L_1 + L_2$ \\
\begin{center}
$\begin{bmatrix}
1 & 0 & 0 & \aug & \frac{4}{52} & \frac{12}{52} & \frac{8}{52} \\
0 & \frac{-1}{4} & 0 & \aug & \frac{5}{52} & \frac{2}{52} & \frac{-3}{52} \\
0 & 0 & 1 & \aug & \frac{1}{13} & \frac{-10}{13} & \frac{2}{13} \\
\end{bmatrix}$
\end{center}

\noindent $L_2 \gets (-4)*L_2$ \\
\begin{center}
\textbf{[$I \vert B^{-1}$]}=$\begin{bmatrix}
1 & 0 & 0 & \aug & \frac{4}{52} & \frac{12}{52} & \frac{8}{52} \\
0 & 1 & 0 & \aug & \frac{-20}{52} & \frac{-8}{52} & \frac{12}{52} \\
0 & 0 & 1 & \aug & \frac{4}{52} & \frac{-40}{52} & \frac{8}{52} \\
\end{bmatrix}$
\end{center}

We have found the same $B^{-1}$ as with the Gauss elimination. Let's test if our solution is right: \\

\begin{center}
$BB^{-1}=\begin{bmatrix}
2 & -2 & 1 \\
1 & 0 & -1 \\
4 & 1 & 1
\end{bmatrix}\frac{1}{52}
\begin{bmatrix}
4 & 12 & 8 & \\
-20 & -8 & 12 \\
4 & -40 & 8
\end{bmatrix} = \frac{1}{52}
\begin{bmatrix}
52 & 0 & 0 \\
0 & 52 & 0 \\
0 & 0 & 52
\end{bmatrix} = I
$
\end{center}

%-------------------------------------------------------------------------------------------------------
\subsection{Exercise 2}
%-------------------------------------------------------------------------------------------------------

\begin{center}
\textbf{B}=$\begin{bmatrix}
2 & -2 & 1 \\
1 & 0 & -1 \\
4 & 1 & 1
\end{bmatrix}$
\end{center}\\

Doolittle's decomposition is obtained through Gauss elimination by storing multipliers in the lower part of B. \\
$L_2 \gets L_2 - \frac{1}{2} L_1 \\
L_3 \gets L_3 - 2L_1 $ \\
\begin{center}
\textbf{B}=
$\begin{bmatrix}
2&-2&1\\
\boxed{\frac{1}{2}}&1&-\frac{3}{2}\\
\boxed{2}&5&-1
\end{bmatrix}
$
\end{center}
$L_3 \gets L_3 - 5L_2 \\$
\begin{center}
\textbf{[L\textbackslash U]}=
$\begin{bmatrix}
2&-2&1\\
\boxed{\frac{1}{2}}&1&-\frac{3}{2}\\
\boxed{2}&\boxed{5}&\frac{13}{2}
\end{bmatrix}
$
\end{center}
Thus,
\begin{center}
\textbf{L}=
$\begin{bmatrix}
1&0&0\\
\frac{1}{2}&1&0\\
2&5&1
\end{bmatrix}
$
\textbf{U}=
$\begin{bmatrix}
2&-2&1\\
0&1&-\frac{3}{2}\\
0&0&\frac{13}{2}
\end{bmatrix}
$
\end{center}

\noindent Now we can solve for $b$: \\
1. Solve $Ly = b$ by forward substitution
\begin{itemize}
\item $y_1 = 1$
\item $\frac{1}{2}y_1 + y_2=2 \Rightarrow y_2=\frac{3}{2}$
\item $2y_1+5y_2+y_3=0 \Rightarrow y_3=-\frac{19}{2}$
\end{itemize}
2. Solve $Ux = y$ by backward substitution
\begin{itemize}
\item $\frac{13}{2}x_3=-\frac{19}{2} \Rightarrow x_3=-\frac{19}{13}$
\item $x_2-\frac{3}{2}x_3=\frac{3}{2} \Rightarrow x_2=-\frac{9}{13}$
\item $2x_1-2x_2+x_3=1 \Rightarrow x_1=\frac{1}{2}(1-\frac{18}{13}+\frac{19}{13})=\frac{7}{13}$
\end{itemize}


%-------------------------------------------------------------------------------------------------------
\subsection{Exercise 3}
%-------------------------------------------------------------------------------------------------------

\begin{center}
\textbf{A}=
$\begin{bmatrix}
1&1&1\\
1&2&2\\
1&2&3
\end{bmatrix}
$
\end{center}

\begin{center}
$\textbf{A}= LL^T =
\begin{bmatrix}
L_{11}&0&0\\
L_{21}&L_{22}&0\\
L_{31}&L_{32}&L_{33}\\
\end{bmatrix}
\begin{bmatrix}
L_{11}&L_{21}&L_{31}\\
0&L_{22}&L_{32}\\
0&0&L_{33}\\
\end{bmatrix}
$
\end{center}

Identification of column 1:
\begin{itemize}
\item $L_{11} = 1$
\item $L_{11}L_{21} = 1 \Rightarrow L_{21} = 1$
\item $L_{11}L_{31} = 1 \Rightarrow L_{31} = 1$
\end{itemize}

Identification of column 2:
\begin{itemize}
\item $L_{21}^2 + L_{22}^2 = 2 \Rightarrow L_{22} = 1$
\item $1+L_{22}L_{32} = 2 \Rightarrow L_{32} = 1$
\end{itemize}

Identification of column 3:
\begin{itemize}
\item $L_{31}^2 + L_{32}^2  + L_{33}^2 = 3 \Rightarrow L_{33} = 1$
\end{itemize}

Therefore,
\begin{center}
\textbf{L}=
$\begin{bmatrix}
1&0&0\\
1&1&0\\
1&1&1
\end{bmatrix}
$
\end{center}

\subsection{Exercise 4}
\textbf{Using Gauss elimination}

\noindent We will solve $DX=I$ by Gauss elimination \\

\begin{center}
\textbf{[$D \vert I$]}=
$\begin{bmatrix}
3&1&2 &\aug&1&0&0\\
1&1&0 &\aug& 0&1&0\\
5&8&9 &\aug& 0&0&1
\end{bmatrix}
$ \\
\end{center}

\noindent $L_2 \leftarrow L_2 - \frac{1}{3}L_1$\\
$L_3 \leftarrow L_3 - \frac{5}{3}L_1$\\

\begin{center}
$\begin{bmatrix}
3&1&2 &\aug&1&0&0\\
0&\frac{2}{3}&-\frac{2}{3} &\aug&-\frac{1}{3}&1&0\\
0&\frac{19}{3}&\frac{17}{3} &\aug& -\frac{5}{3}&0&1
\end{bmatrix}
$ \\
\end{center}

$L_3 \leftarrow L_3 - \frac{19}{2}L_2$\\

\begin{center}
$\begin{bmatrix}
3&1&2 &\aug&1&0&0\\
0&\frac{2}{3}&-\frac{2}{3} &\aug&-\frac{1}{3}&1&0\\
0&0&12 &\aug& \frac{3}{2}&-\frac{19}{2}&1
\end{bmatrix}
$ \\
\end{center}

Back substitution
\begin{itemize}
\item $12x_{31} = \frac{3}{2} \Rightarrow x_{31} = -\frac{1}{8}$
\item $\frac{2}{3}x_{21} = -\frac{1}{3} + \frac{2.1}{3.8} \Rightarrow x_{21} = -\frac{3}{8}$
\item $x_{11} = \frac{1}{3}(1+\frac{3}{8}-\frac{2}{8}) = \frac{3}{8}$
\end{itemize}

\begin{itemize}
\item $12x_{32} = -\frac{19}{2} \Rightarrow x_{32} = -\frac{19}{24}$
\item $x_{22} = -\frac{3}{2}(1-\frac{2}{3}.\frac{19}{24}) = -\frac{17}{24}$
\item $x_{12} = \frac{1}{3}(0-\frac{17}{24}+2\frac{19}{24}) = \frac{7}{24}$
\end{itemize}

\begin{itemize}
\item $12x_{33} = 1 \Rightarrow x_{33} = \frac{1}{12}$
\item $x_{23} = -\frac{3}{2}(0+\frac{2}{3}.\frac{1}{12}) = \frac{1}{12}$
\item $x_{13} = \frac{1}{3}(0-\frac{1}{12}-\frac{2}{12}) = -\frac{1}{12}$
\end{itemize}

Finally,
\begin{center}
$C^{-1} = \begin{bmatrix}
\frac{3}{8} & \frac{7}{24} & \frac{-1}{12} \\
\frac{-3}{8} & \frac{17}{24} & \frac{1}{12} \\
\frac{1}{8} & \frac{-19}{24} & \frac{1}{12} \\
\end{bmatrix}
$
\end{center}

\end{document}
